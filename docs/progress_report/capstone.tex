\documentclass[onecolumn, draftclsnofoot,10pt, compsoc]{IEEEtran}
\usepackage{graphicx}
\usepackage{url}
\usepackage{setspace}

\usepackage{geometry}
\geometry{textheight=9.5in, textwidth=7in}

% 1. Fill in these details
\def \CapstoneTeamName{		SmartRC}
\def \CapstoneTeamNumber{		42}
\def \GroupMemberOne{			Michael Burton}
\def \GroupMemberTwo{			Rajat Kulkarni}
\def \GroupMemberThree{			Logan Saso}
\def \GroupMemberFour{			Miao Zhou}
\def \CapstoneProjectName{		Autonomous RC Car}
\def \CapstoneSponsorPerson{		D. Kevin McGrath}

% 2. Uncomment the appropriate line below so that the document type works
\def \DocType{		%Problem Statement
				%Requirements Document
				%Technology Review
				%Design Document
				Progress Report
				}
			
\newcommand{\NameSigPair}[1]{\par
\makebox[2.75in][r]{#1} \hfil 	\makebox[3.25in]{\makebox[2.25in]{\hrulefill} \hfill		\makebox[.75in]{\hrulefill}}
\par\vspace{-12pt} \textit{\tiny\noindent
\makebox[2.75in]{} \hfil		\makebox[3.25in]{\makebox[2.25in][r]{Signature} \hfill	\makebox[.75in][r]{Date}}}}
% 3. If the document is not to be signed, uncomment the RENEWcommand below
%\renewcommand{\NameSigPair}[1]{#1}

%%%%%%%%%%%%%%%%%%%%%%%%%%%%%%%%%%%%%%%
\begin{document}
\begin{titlepage}
    \pagenumbering{gobble}
    \begin{singlespace}
    	\includegraphics[height=4cm]{coe_v_spot1}
        \hfill 
        % 4. If you have a logo, use this includegraphics command to put it on the coversheet.
        %\includegraphics[height=4cm]{CompanyLogo}   
        \par\vspace{.2in}
        \centering
        \scshape{
            \huge CS Capstone \DocType \par
            {\large\today}\par
            \vspace{.5in}
            \textbf{\Huge\CapstoneProjectName}\par
            \vfill
            {\large Prepared for}\par
            \vspace{5pt}
            {\Large\NameSigPair{\CapstoneSponsorPerson}\par}
            {\large Prepared by }\par
            Group\CapstoneTeamNumber\par
            % 5. comment out the line below this one if you do not wish to name your team
            \CapstoneTeamName\par 
            \vspace{5pt}
            {\Large
                \NameSigPair{\GroupMemberOne}\par
                \NameSigPair{\GroupMemberTwo}\par
                \NameSigPair{\GroupMemberThree}\par
                \NameSigPair{\GroupMemberFour}\par
            }
            \vspace{20pt}
        }
        \begin{abstract}
        % 6. Fill in your abstract    
  This document provides an overview of the tasks we completed, and the progress we have made for our project during the fall term. This document includes a week by week breakdown of what we did, as well as a general overview of our progress.

        \end{abstract}     
    \end{singlespace}
\end{titlepage}
\newpage
\pagenumbering{arabic}
\tableofcontents
% 7. uncomment this (if applicable). Consider adding a page break.
%\listoffigures
%\listoftables
\clearpage



\section{Project Recap}
The purpose of this project is to produce an autonomous driving RC car that is able to solve many of the same problems full-scale autonomous vehicles are currently having to work on such as classifying objects, detecting static and dynamic obstacles, and making the proper movement based on the determined situation.

Our main goal is to implement a fully autonomous RC car that can navigate a busy room without collisions. The car is equipped with Ultrasonic, LiDAR, and optical sensors in order to provide the proper data. The car will need to travel between predetermined points A and B safely. It will also need to avoid static objects, as well as to predict and avoid moving obstacles. This car should also be able to navigate extreme circumstances such as quickly appearing obstacles. The testing criteria will be to instruct the car to drive down a road and return back to the starting point.


\section{Current Progress}
Currently, we are almost done designing the project and are ready to start implementation. Once we actually get access to the hardware materials during Winter Term we should be able to fill in any missing gaps for the design section.  

We have determined what hardware our car will be using based on our research and needs for our software system. We have also determined what software tools we will be using for the project. Some examples of the types of hardware that will be utilized are the LiDAR, Ultrasonic and Optical sensors, as well as the Traxxas car body. We have a general idea about the software that we'll be using to implement the car. However, without much hands-on experience it's hard to tell if the choices we made are the correct ones. An example of the types of software we'll be using is OpenCV.

We have come up with general design decisions. For example, we have figured out how to utilize the sensors to ensure that the RC car knows when an object is moving towards it at different angles. Furthermore, we have also come up with various testing methods that will be implemented to gauge the effectiveness of the RC car.


\section{Problems Encountered}
One of the problems we have encountered is our group's general lack of experience when it comes to working with machine learning. We have started to work on this issue by having each member better familiarize themselves with the topic over winter break. 

We have also had issues with getting consistent or timely responses from our client. For the last three weeks of the term we have been unable to contact our client.

For the most part we've already been told what pieces of hardware we're using. As a result, our research focused more on machine learning platforms to use. With the materials we've been told we'll use, we found a few pieces of software that come with the device that should prove to be more than useful. However, we can't test without the hardware which has yet to be given to us.

Besides this, everything has been progressing smoothly and consistently. We have been able to finish assignments in a timely manner.


\section{Retrospective}
%Week One
\def \wkOneP {
Week 1: we were each presented with many interesting projects, to individually choose from.
}
\def \wkOneD {
The autonomous R/C car project was assigned to each of us, and our team was formed.
}
\def \wkOneA {
We listed our top project choices and were assigned to this project. 
}

%Week Two
\def \wkTwoP {
Week 2: We met up as a group for the first time, introduced ourselves, and discussed the project for the first time. Additionally, we were able to get into contact with our client via video call.  
}
\def \wkTwoD {
Our client provided us with information about the hardware we are supposed to be using and suggested we research ultrasonic, LiDAR, and optical sensors.
We were also tasked with each creating our own individual problem statement for the project.
We were also tasked with making a requirements document.

}
\def \wkTwoA {
We started working on the requirements document and researching the hardware our client referred us to. 
}

%Week Three
\def \wkThreeP {
Week 3: Our group met to discuss our Team Standards document, Requirements document, and a rough timeline for our project.
}
\def \wkThreeD {
We were tasked with creating a group problem statement, as well as the first draft of our requirements document. We also decided to create a rough schedule for the rest of the term.
}
\def \wkThreeA {
We finished the first draft of our requirements document. We also finished our group problem statement. A rough draft of our project’s timeline was also produced. 
}

%Week Four
\def \wkFourP {
Week 4: We received feedback for our requirements document draft. We also started discussing the software side of our project. 
}
\def \wkFourD {
We needed to revise our requirements document to address the issues presented to us in our peer review. 
}
\def \wkFourA {
We finished the final draft of our requirements document and started doing research into what solutions major car companies used for their autonomous driving systems.
}

%Week Five
\def \wkFiveP {
Week 5: We met with our TA to get a better understanding of the structure the technical document should follow.
}
\def \wkFiveD {
Each member was tasked with creating a technical review covering three unique subtopics in the scope of our project.
}
\def \wkFiveA {
We divided the pieces of our project so that each group member had three technical topics to discuss in their tech review. We then finished the first draft of our tech review.
}

%Week Six
\def \wkSixP {
Week 6: We asked our TA to review our tech review document to provide us with useful feedback for finalizing our draft.
}
\def \wkSixD {
We needed to revise our technical review document based on the feedback we received from our TA.
}
\def \wkSixA {
We finalized the tech review document and tried to get into contact with our client via email.
}

%Week Seven
\def \wkSevenP {
Week 7: We met with our TA to discuss the design document in greater detail. This gave us a better understanding of how the design document should look. 
}
\def \wkSevenD {
We were given the task of creating the design document for our project. 
}
\def \wkSevenA {
We tried to get into contact with our client for the betterment of our design document but were unable to. We finished the first draft of our design document.
}

%Week Eight
\def \wkEightP {
Week 8: We met with our TA and further discussed the design document.
}
\def \wkEightD {
We needed to continue researching and coming up with ideas and solutions to put inside of our design document.
}
\def \wkEightA {
We continued to refine our design document. We also again tried to get in contact with our client again but were unable to reach them.
}

%Week Nine
\def \wkNineP {
Week 9: We communicated remotely about the design document since we were out of town for thanksgiving.
}
\def \wkNineD {
The design document needed to be finished.
}
\def \wkNineA {
We finished and submitted the design document. We tried to contact our client again but were still unable to make contact.
}

%Week Ten
\def \wkTenP {
Week 10: We had our final meeting with our TA where we discussed the progress report and what state our project was in before the end of the term.
}
\def \wkTenD {
We were tasked with creating a progress report, as well as compiling an email with our problem statement, requirements document, and design document to send to our client for verification.
}
\def \wkTenA {
After compiling all of the required documents for the term, we emailed our client to receive approval. We also finished reflection of the entire term.
}
\begin{table}[h]
\begin{tabular}{|p{0.3\linewidth}|p{0.3\linewidth}|p{0.3\linewidth}|}
\hline
Positive      & Delta      & Action    \\ \hline
\wkOneP       & \wkOneD    & \wkOneA   \\ \hline
\wkTwoP       & \wkTwoD    & \wkTwoA   \\ \hline
\wkThreeP     & \wkThreeD  & \wkThreeA \\ \hline
\wkFourP      & \wkFourD   & \wkFourA  \\ \hline
\wkFiveP      & \wkFiveD   & \wkFiveA  \\ \hline
\wkSixP       & \wkSixD    & \wkSixA   \\ \hline
\wkSevenP     & \wkSevenD  & \wkSevenA \\ \hline
\wkEightP     & \wkEightD  & \wkEightA \\ \hline
\wkNineP      & \wkNineD   & \wkNineA  \\ \hline
\wkTenP       & \wkTenD    & \wkTenA   \\ \hline
\end{tabular}
\end{table}

\end{document}